This sounds like a very complex question -- and it is -- but we can use chi-square tests to study an important aspect of the problem. We will label each day as \resp{Up} or \resp{Down} depending on whether market was up or down. This simplifies our problem: instead of having a sequence of stock changes, we have a sequence of Ups and Downs (D), as shown below:
\begin{center}\footnotesize
\begin{tabular}{lc ccc ccc ccc cc}
Price		&	& \footnotesize2.52 &
	\footnotesize-1.46 & \footnotesize 0.51 &
	\footnotesize-4.07 & \footnotesize3.36 &
	\footnotesize1.10 &
	\footnotesize-5.46 & \footnotesize-1.03 & \footnotesize-2.99 & \footnotesize1.71 \\
Outcome	 & \hspace{0mm} &
	\textbf{Up} &
	D & \textbf{Up} &
	D & \textbf{Up} &
	\textbf{Up} &
	D & D & D & \textbf{Up} \\
\footnotesize Days to \resp{Up} & & 1 & - & 2 & - & 2 & 1 & - & - & - & 4 \\
\end{tabular}
\end{center}
If the days really are independent, then the number of days until a positive trading day should follow a geometric distribution\footnote{Section~3.2 would be useful background reading for this example, but it is not a prerequisite.}. In the data above, it took only one day until the market was up, so the first wait time was 1 day. It took two more days before we observed our next \resp{Up} trading day, and two more for the third \resp{Up} day. We would like to determine if these counts -- 1, 2, 2, 1, 4, and so on -- follow the geometric distribution. Table~\ref{sAndP500For1990To2009TimeToPosTrade} shows the number of waiting days for a positive trading day during 1990-2009 for the S\&P500.
\begin{table}[h]
\begin{center}
\begin{tabular}{ll ccc ccc c ll}
\hline
Days	 & \hspace{2mm} & 1 & 2 & 3 & 4 & 5 & 6 & 7+ & \hspace{2mm} & Total \\
Occurrences &		& 1298 & 685 & 367 & 157 & 77 & 33 & 20 & & 2587 \\
\hline
\end{tabular}
\end{center}
\caption{Distribution of the waiting time until a positive trading day.} % The data comes from 1990-2009 for the S\&P500 stock index.} % (S\&P500, 1990-2009).}
\label{sAndP500For1990To2009TimeToPosTrade}
\end{table}

